%----------------------------------------------------
% Basic Information: Location, TA Info, Contact Info
%----------------------------------------------------
\begin{tabular}{l p{0.65\linewidth}}

\textbf{Lecture Times:}         & Tue--Thur 2:00pm--3:20pm \\

\textbf{Lecture Location:}      & KPTC 309 (Kersten Physics Teaching Center) \\

\textbf{Textbook:}              & None required, many suggested! (see Reading List) \\

\textbf{Canvas Course Site:}    & \CanvasLink \\

\textbf{Piazza Site:}           & \href{https://piazza.com/uchicago/spring2018/phys227}{https://piazza.com/uchicago/spring2018/phys227} \\

\textbf{Prof. Office Hours:}    & Mon. 2:00pm--3:00pm, and by appt. \\

\textbf{Computer Lab Hours:}    & Help is available specifically for PHYS 250 by TAs in the Computer Science Instructional Lab (CSIL), 1st floor of Crerar, on: \\
                                & Tue 7:00-9:00 pm in CSIL 1 \\
                                & Wed 2:30-4:00 pm in CSIL 2 \\
                                & Wed 7:00-9:00 pm in CSIL 1 \\ 

\textbf{Teaching Assistants:}   & See the \textit{Additional Information} section.\\
 
\textbf{Description:}           & This course introduces the use of computers in the physical sciences. 
                                  After an introduction to programming basics, we will cover numerical solutions 
                                  to fundamental types of problems, techniques for manipulating large data
                                  sets, neural networks, and the basics of data analysis. \\

 
\textbf{Homework (70\%):}       &  \emph{\textbf{Canvas:}} Problem sets and materials 
                                   are available on \href{\CanvasURL}{Canvas}. \\
                                & \emph{\textbf{Due Date:}} \textbf{\underline{Fridays}}. 
                                  Graded problem sets will returned during Discussion
                                  (afterwards in envelopes near KPTC 106). \\
                                & \emph{\textbf{Collaboration Policy:}} Collaboration on issues,
                                  concepts, and approaches is encouraged, but the work
                                  \textit{must be your own}.\\

\textbf{Final Project (30\%):}  & Week of Dec. 3rd (\textbf{\it Details TBD}) \\

\end{tabular}