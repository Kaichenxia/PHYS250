%----------------------------------------------------
% Piazza
%----------------------------------------------------
\noindent \textbf {\Large \sc Additional Information for the Course}:

\vspace{1cm}

%----------------------------------------------------
\noindent \underline{Teaching Assistants (TA):}
%----------------------------------------------------

\noindent TA's will have office hours as well as be available in the CSIL lab for assistance.

\textbf{Office Hours:}

\begin{tabular}{l l | l | l }

TA Name & Email & Office Hours & Location \\ \hline

Jack Dale          & \href{mailto:jdale1@uchicago.edu}{jdale1@uchicago.edu}  & FRI  3:00-4:00 pm  & KPTC 307 \\
Jan Offermann      & \href{mailto:jano@uchicago.edu}{jano@uchicago.edu}      & TUE 12:30-1:30 pm  & KPTC 307 \\


\end{tabular}

\vspace{1cm}

\textbf{CSIL Lab Assistance}

\begin{tabular}{l l | l | l }

TA Name & Email & Office Hours & CSIL Lab \\ \hline

Jan Offermann          & \href{mailto:jdale1@uchicago.edu}{jdale1@uchicago.edu}  & TUE  7:00-9:00 pm  & CSIL 1\&2 \\
Jan Offermann      & \href{mailto:jano@uchicago.edu}{jano@uchicago.edu}      & WED  2:30-3:50 pm  & CSIL 1\&2 \\
Jack Dale     & \href{mailto:jano@uchicago.edu}{jano@uchicago.edu}      & WED  7:00-9:00 pm  & CSIL 1\&2 \\


\end{tabular}

\vspace{1cm}

%%----------------------------------------------------
%\noindent \underline{Exams:}
%%----------------------------------------------------
%  
%  \begin{itemize}
%  
%    \item There will be two exams, a mid-term given during the lecture period as scheduled in the syllabus and a final exam, as scheduled by the Registrar.
%    \item The mid-term will be given on Monday, Apr. 30 during class.  The mid-term will count toward 25\% of your grade.
%    \item The final exam will be given on Wednesday, June 6 from 10:30am--12:30pm. The final exam will count toward 30\% of the course grade.
%    \item The exams will be closed book. For both exams, you will be given supplementary material with the exam. Calculators will be permitted for use during the exam. No other notes or references will be allowed.
%    \item \textbf{\emph{There are no make-up exams}}. If there is an important reason that you cannot take an exam, make arrangements with the Prof. David Miller at least one day in advance of the exam. If you are sick or an emergency arises which prevents you from taking the exam without prior arrangement, you must have a signed note from your doctor or undergraduate advisor explaining your absence. No exceptions will be made to this rule and you will get a 0 grade for a missed exam if these conditions are not met.
%
%  \end{itemize}  
%  
%
%\newpage  
%----------------------------------------------------
\noindent \underline{Schedule and Section Assignment Information:}
%----------------------------------------------------

  \begin{itemize}

    \item CSIL Lab Sections will begin meeting in Week 2 of Spring Quarter. 

  \end{itemize}
  


%----------------------------------------------------
\noindent \underline{Piazza}
%----------------------------------------------------

\noindent This term we will be using \textbf{Piazza} for class discussion. The system is highly catered to getting you help fast and efficiently from classmates, the TA, and myself. 

\begin{itemize}

  \item Rather than emailing questions to the teaching staff, I encourage you to post your questions on Piazza. 

  \item If you have any problems or feedback, email myself or the Piazza developers: \href{mailto:team@piazza.com}{team@piazza.com}

  \item Find our class page at: \PiazzaLink
  
\end{itemize}  


%----------------------------------------------------
\noindent \underline{Final Poster Project}
%----------------------------------------------------

\begin{itemize}

  \item \textbf{Topic:} The topic should fall within the class's theme of modeling and analyzing physical systems. A complete project will include software in a GitHub repository written in a language of your choice that can be run with minimal setup by the Instructor and TAs. This project should explore a phenomenon (or phenomena) that are only properly or fully analyzable with computational methods, or for which we learn significantly more than with analytic methods alone. The proposed topic must be selected by Tuesday of 5th Week (Oct 29, 2019) in consultation with the instructor and TA, although modifications and changes are certainly allowed, but are strongly encouraged to be discussed first. You are encouraged to include a sketch of the poster (see below), a description of the equations you plan to solve or analyze, and the primary results or even figures that you plan to include.

  \item \textbf{Structure:} The projects are individual and are not allowed to be done jointly with others. However, you are strongly encouraged to discuss approaches, methods, ideas, implementations, and results with any and everyone.

  \item \textbf{Poster Presentation:} The project will be presented in a poster session at the end of the term on the Friday of Reading Period. All posters will be hung simultaneously and audience members will be circulating to review the posters. Your poster will need to be printed the day \textbf{before} at the very latest. The project's poster will be evaluated by at least 2 judges (including the TAs and instructor). The criteria and rubric will be distributed in advance to the class. In addition, the GitHub repository must be indicated on the poster, and it will be tested separately for its ability to be executed, as well as for its design and implementation of the key computational methods used. 
  
\end{itemize}  

